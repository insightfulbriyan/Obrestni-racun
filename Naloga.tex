\documentclass[12pt]{article}


\usepackage[
    a4paper, 
    margin=2.5cm]{geometry}
\usepackage[utf8]{inputenc}         % UTF8 enkodiranje
\usepackage[slovene]{babel}         % Slovenščina
\usepackage[
    pdfusetitle, 
    hidelinks, 
    unicode]{hyperref}              % Nastavi atribute PDF-ja, ne označuj povezav
\usepackage{microtype}              % Izboljšave za tipografijsko perfekcijo :)
\usepackage{enumitem}               % Seznami za člene
\usepackage{graphicx}               % Vključitev slik
\usepackage{dirtytalk}              % Citat
\usepackage{listings}               % Kodni blok
\usepackage{fancyvrb}
\usepackage[font=]{caption}         % Required for specifying captions
\usepackage[normalem]{ulem}         % Krašanje enot v enačbi
\usepackage{times}                  % Times New Roman pisava
\usepackage{tikz} 
\usepackage[european]{circuitikz}   % Električna vezja
\usepackage{datetime}               % Datum
% \usepackage[slovenian]{csquotes}
\usepackage{braket}
\usepackage{amsmath} % matematika ki izgleda lepo
\usepackage{amsfonts} % množice
\usepackage[style=ieee, maxbibnames=3, minbibnames=1, 
    maxcitenames=1, mincitenames=1, sorting=nyt]{biblatex}   % Navajanje virov
\bibliography{viri}

\urlstyle{rm}

\setlength{\parindent}{0em}
\setlength{\parskip}{1ex}

\setcounter{secnumdepth}{5}
\setcounter{tocdepth}{4}

\renewcommand{\thesection}{\arabic{section}}
\renewcommand{\thesubsection}{\thesection.\arabic{subsection}}
\renewcommand{\thesubsubsection}{\thesubsection.\arabic{subsubsection}}
\renewcommand{\theparagraph}{\thesubsubsection.\arabic{paragraph}}
\renewcommand{\thesubparagraph}{\theparagraph.\arabic{subparagraph}}

\renewcommand{\labelnamepunct}{\addcomma\space}
\DeclareFieldFormat[article]{title}{#1}
\DeclareFieldFormat[online]{title}{\mkbibemph{#1}}

\DefineBibliographyStrings{slovene}{
  andothers = {et. al\adddot},
  urlseen = {dostopano:}
}

\newdateformat{MMYYYYdate}{\monthname[\THEMONTH] \THEYEAR}

\title{Obrestni račun}
\author{Jaka Kovač, G 4. b}

\begin{document}
\pagenumbering{arabic}

\begin{center}
    \thispagestyle{empty}
    \includegraphics[scale=1]{slike/logotip_vegova_leze_brezokvirja.png}
    \\
    \textbf{Vegova ulica 4, 1000 Ljubljana}

    \vspace{\fill} 
    Seminarska naloga pri predmetu matematika

    \Huge{\textbf{Obrestni račun}}

    \normalsize
    \vspace{\fill}

    Mentor: Karin Kastelic, prof. mat., spec. \hfill Avtor: Jaka Kovač, G 4. b\\
    \null
    Ljubljana, oktober 2023 – \MMYYYYdate\today 
\end{center}
\newpage
\thispagestyle{empty}
\null
\newpage

\section*{Povzetek}
V tej seminarski nalogi bom predstavil obrestni rečun in njegove vrste ter primere uporabe.
\\ %prazna vrstica
\textbf{Ključe besede:} obrestni račun, obrestna mera, anuiteta, amortizacijski načrt

\vfill
\section*{Abstract}
\foreignlanguage{english}{This paper describes mathematics behind interest rates and their usecases.
\\ %prazna vrstica
\textbf{Keywords:} interest rate, annuity, amortization schedule}
\vfill

% KAZALO 
\newpage
\thispagestyle{empty} % ne številčimo strani
\tableofcontents % kazalo

% \begingroup     % kazalo slik
% \makeatletter
% \section*{Slike}
% \@starttoc{lof}
% \let\clearpage\relax
% \makeatother
% \endgroup


\newpage
\section{Uvod}
Že od nekdaj so ljudje med seboj trgovali. Včasih so med sabo menjali dobrine (naturalno gospodarstvo),
ko pa so okoli leta 3000 pr. n. št. v Mezopotamiji \cite{wiki:money} pričeli z menjavo izdelov za denar.
Izumu denarja so botrovale tudi banke. Posojanje denarja v zameno za več denarja se sprva zdi precej
nenavadno, vendar pa je to le ena izmed storitev modernejšega sveta. 



\section{Teorija}
    \subsection{Pojmi, definicije in uporabljeni simboli}
        \begin{table}[h!]
            \centering
            \begin{tabular}{|c|c|p{9cm}|}
                \hline
                \textbf{simbol} & \textbf{pojem} & \textbf{definicija} \\ \hline
                $G_0$ & glavnica & denarna vrednost, ki si jo od nekoga izposodimo ali jo mi posodimo nekomu \\ \hline 
                $o$ & obresti & nadomestilo ali odškodnina za izposojeni denar, ker le-ta v času obrestovanja ni na voljo lastniku \\ \hline
                $p$ & obrestna mera & obresti podane v odstodkih navadno \textbf{letna obrestna mera} \\ \hline
                $p_k$ & konformna obrestna mera & obrestna mera, ki se uporablja za izračun obresti \\ \hline
                $r$ & obrestovalni faktor & $$r = 1 + \frac{p}{100}$$ \\ \hline
                $r_k$ & konformni obrestovalni faktor & $$r_k = 1 + \frac{p_k}{100}$$ \\ \hline
                $k$ & kapitalizaijska obdobja & \\ \hline
                $c$ & anuiteta & redno odplačilo \\ \hline
            \end{tabular}

            \medskip
            \centering povteto po \cite{vega4}
            \caption{Simboli, pojmi in njihove definicije}
            \label{tab:simboli}
        \end{table}
    \subsection{Obrestovanje}
        \subsubsection{Navadno obrestovanje}
        Navadno obrestovanje je način obrestovanja, kjer so obresti odvisne le od glavnice in obrestne mere,
        ne pa tudi od prejšnjih obresti. Končna vrednost glavnice po $n$ letih se izračuna po formuli:
        \begin{equation}
            G_n = G_0 \cdot (1 + \frac{p*n}{100})
        \end{equation}

        \newpage
        Vzemimo primer, kjer na banko položimo 10 000 € za 5 let pri 5\% letni obrestni meri. 
        \footnote{Ničto leto označuje polog denarja}
        \begin{center}
            
            \begin{tabular}{|c|c|}
                \hline
                \textbf{leto} & \textbf{vrednost} \\ \hline
                0 & 10 000 € \\ \hline
                1 & 10 500 € \\ \hline
                2 & 11 000 € \\ \hline
                3 & 11 500 € \\ \hline
                4 & 12 000 € \\ \hline
                5 & 12 500 € \\ \hline
                
            \end{tabular}
        \end{center}
        \subsubsection{Obrestno obrestovanje}

\section{Rešen primer}
\section{Avtentični primer}






\newpage
\begingroup
\makeatletter
    \section{Viri in literatura}
    \nocite{*}
    \printbibliography[heading=none]
\makeatother
\endgroup
\newpage

\begin{samepage}
    \thispagestyle{empty}
    \section*{Izjava o avtorstvu}
    Izjavljam, da je seminarska naloga v celoti moje avtorsko delo, ki sem ga 
    izdelal samostojno s pomočjo navedene literature in pod vodstvom mentorja.

    \vfill
    
    \today \hfill Jaka Kovač, G 4. b
    
    \vspace{3 cm}
\end{samepage}

\end{document}
