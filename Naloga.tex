\documentclass[12pt]{article}


\usepackage[
    a4paper, 
    margin=2.5cm]{geometry}
\usepackage[utf8]{inputenc}         % UTF8 enkodiranje
\usepackage[slovene]{babel}         % Slovenščina
\usepackage[
    pdfusetitle, 
    hidelinks, 
    unicode]{hyperref}              % Nastavi atribute PDF-ja, ne označuj povezav
\usepackage{microtype}              % Izboljšave za tipografijsko perfekcijo :)
\usepackage{enumitem}               % Seznami za člene
\usepackage{graphicx}               % Vključitev slik
\usepackage{dirtytalk}              % Citat
\usepackage{listings}               % Kodni blok
\usepackage{fancyvrb}
\usepackage[font=]{caption}         % Required for specifying captions
\usepackage[normalem]{ulem}         % Krašanje enot v enačbi
\usepackage{times}                  % Times New Roman pisava
\usepackage{tikz} 
\usepackage[european]{circuitikz}   % Električna vezja
\usepackage{datetime}               % Datum
\usepackage{siunitx}                % tabele
% \usepackage[slovenian]{csquotes}
\usepackage{braket}
\usepackage{amsmath}                % matematika ki izgleda lepo
\usepackage{amsfonts}               % množice
\usepackage{longtable}              % več vrstic v tabeli
\usepackage[style=ieee, maxbibnames=3, minbibnames=1, 
    maxcitenames=1, mincitenames=1, sorting=nyt]{biblatex}   % Navajanje virov
\bibliography{viri}

\urlstyle{rm}

\setlength{\parindent}{0em}
\setlength{\parskip}{1ex}

\setcounter{secnumdepth}{5}
\setcounter{tocdepth}{4}

\renewcommand{\thesection}{\arabic{section}}
\renewcommand{\thesubsection}{\thesection.\arabic{subsection}}
\renewcommand{\thesubsubsection}{\thesubsection.\arabic{subsubsection}}
\renewcommand{\theparagraph}{\thesubsubsection.\arabic{paragraph}}
\renewcommand{\thesubparagraph}{\theparagraph.\arabic{subparagraph}}

\renewcommand{\labelnamepunct}{\addcomma\space}
\DeclareFieldFormat[article]{title}{#1}
\DeclareFieldFormat[online]{title}{\mkbibemph{#1}}

\DefineBibliographyStrings{slovene}{
  andothers = {et. al\adddot},
  urlseen = {dostopano:}
}

\newdateformat{MMYYYYdate}{\monthname[\THEMONTH] \THEYEAR}

\title{Obrestni račun}
\author{Jaka Kovač, G 4. b}

\begin{document}
\pagenumbering{arabic}

\begin{center}
    \thispagestyle{empty}
    \includegraphics[scale=1]{slike/logotip_vegova_leze_brezokvirja.png}
    \\
    \textbf{Vegova ulica 4, 1000 Ljubljana}

    \vspace{\fill} 
    Seminarska naloga pri predmetu matematika

    \Huge{\textbf{Obrestni račun}}

    \normalsize
    \vspace{\fill}

    Mentor: Karin Kastelic, prof. mat., spec. \hfill Avtor: Jaka Kovač, G 4. b\\
    \null
    Ljubljana, oktober 2023 – \MMYYYYdate\today 
\end{center}
\newpage
\thispagestyle{empty}
\null
\newpage

\section*{Povzetek}
V tej seminarski nalogi bom predstavil obrestni rečun in njegove vrste ter primere uporabe.
\\ %prazna vrstica
\textbf{Ključe besede:} obrestni račun, obrestna mera, anuiteta, amortizacijski načrt

\vfill
\section*{Abstract}
\foreignlanguage{english}{This paper describes mathematics behind interest rates and their
usecases.
\\ %prazna vrstica
\textbf{Keywords:} interest rate, annuity, amortization schedule}
\vfill

% KAZALO 
\newpage
\thispagestyle{empty} % ne številčimo strani
\tableofcontents % kazalo

% \begingroup     % kazalo slik
% \makeatletter
% \section*{Slike}
% \@starttoc{lof}
% \let\clearpage\relax
% \makeatother
% \endgroup


\newpage
\section{Uvod}
Že od nekdaj so ljudje med seboj trgovali. Včasih so med sabo menjali dobrine (naturalno 
gospodarstvo), ko pa so okoli leta 3000 pr. n. št. v Mezopotamiji \cite{wiki:money} pričeli
z menjavo izdelov za denar. Izumu denarja so botrovale tudi banke. Posojanje denarja v 
zameno za več denarja se sprva zdi precej nenavadno, vendar pa je to le ena izmed storitev
modernejšega sveta. 

\section{Teorija}
    \subsection{Pojmi, definicije in uporabljeni simboli}
        \begin{table}[h!]
            \centering
            \begin{tabular}{|c|c|c|p{7cm}|}
                \hline
                \textbf{simbol} & \textbf{pojem} & \textbf{enota} & \textbf{definicija} \\ \hline
                $G_0$ & glavnica                & EUR    & denarna vrednost, ki si jo od nekoga izposodimo ali jo mi posodimo nekomu \\ \hline 
                $o$   & obresti                 & EUR    & nadomestilo ali odškodnina za izposojeni denar, ker le-ta v času obrestovanja ni na voljo lastniku \\ \hline
                $p$   & obrestna mera           & \%     & obresti podane v odstodkih navadno \textbf{letna obrestna mera} \\ \hline
                $p_k$ & konformna obrestna mera & \%     & obrestna mera, ki se uporablja za izračun obresti \\ \hline
                $r$   & obrestovalni faktor     &        & $$r = 1 + \frac{p}{100}$$ \\ \hline
                $r_k$ & konformni obrestovalni faktor &  & $$r_k = 1 + \frac{p_k}{100}$$ \\ \hline
                $k$   & število kapitalizacijskih odbobji &  & kolikokrat smo izračunali obresti \\ \hline
                $c$   & anuiteta                & EUR    & redno odplačilo \\ \hline
            \end{tabular}

            \medskip
            \centering povteto po \cite{vega4}
            \caption{Simboli, pojmi in njihove definicije}
            \label{tab:simboli}
        \end{table}

        \subsubsection{Prikazovanje podatkov v tabelah in izračunih}
        Zaradi preprostosti prikazavanja so v tabeli vrednosti zaokrožene na cente natačno
        in prikazane v EUR. V izračunih se uporablja dejanska vrednost. Ničto leto označuje
        polog denarja, prikazane vrednosti pa stanje ob koncu leta razen kjer je navedeno 
        drugače. 
    \subsection{Obrestovanje}
        \subsubsection{Navadno obrestovanje}
        Navadno obrestovanje je način obrestovanja, kjer so obresti odvisne le od glavnice
        in obrestne mere, ne pa tudi od prejšnjih obresti. Končna vrednost glavnice po $n$
        letih se izračuna po formuli:
        \begin{equation}
            G_n = G_0 \cdot (1 + \frac{p*n}{100})
        \end{equation}

        Vzemimo primer, kjer na banko položimo 10 000 € za 5 let pri 5\% letni obrestni meri. 
        \begin{center}
            
            \begin{tabular}{|c|c|}
                \hline
                \textbf{leto} & \textbf{vrednost [EUR]} \\ \hline
                0 & 10 000,00 \\ \hline
                1 & 10 500,00 \\ \hline
                2 & 11 000,00 \\ \hline
                3 & 11 500,00 \\ \hline
                4 & 12 000,00 \\ \hline
                5 & 12 500,00 \\ \hline
            \end{tabular}
        \end{center}

        \subsubsection{Obrestno obrestovanje}
        Obrestno obrestovanje je način obrestovanja, kjer so obresti odvisne tako od 
        glavnice, obrestne mere in prejšnjih obresti. Vsota glavnice in obresti torej 
        postne glavnica za naslednje kapitalizacijsko obdobje. Končna vrednost glavnice 
        po $n$ letih se izračuna po formuli:
        \begin{equation}
            G_n = G_0 \cdot r^n
        \end{equation}

        Obrestovalni faktor $r$ izračunamo po formuli:
        \begin{equation}
            r = 1 + \frac{p}{100}
        \end{equation}
        
        Enotna formula za izrčun vrdnosti glavnice:
        \begin{equation}
            G_n = G_0 \cdot (1 + \frac{p}{100})^n
        \end{equation}


        Za primer vzemimo enake podatke kot pri navadnem obrestovanju.
        \begin{center}
            \begin{tabular}{|c|S|S|S|}
                \hline
                \textbf{leto} & \textbf{letne obresti [EUR]} & \textbf{skupne obresti [EUR]} & \textbf{vrednost [EUR]} \\ \hline
                0 & 0,00 & 0,00 & 10 000,00 \\ \hline
                1 & 500,00 & 500,00 & 10 500,00 \\ \hline
                2 & 525,00 & 1 025,00 & 11 025,00 \\ \hline
                3 & 551,25 & 1 576,25 & 11 576,25 \\ \hline
                4 & 578,81 & 2 155,06 & 12 155,06 \\ \hline
                5 & 636,69 & 2 762,82 & 12 762,82 \\ \hline
            \end{tabular}
        \end{center}

    \subsection{Obrestna mera}
        \subsubsection{Relativna obrestna mera}
        Če je v enem letu več kapitalizacijskih obdobji, nam pa je podana letna obrestna 
        mera si lahko s formulo \eqref{rom} izračunamo obrestno mero, ki se dejansko 
        uporabi za izračun obresti na koncu vsakega kapitalizacijskega obdobja.

        \begin{equation}
            p(k) = \frac{p(letna)}{k}
            \label{rom}
        \end{equation}

        Formula za obrestovalni faktor je:
        \begin{equation}
            \begin{split}
                r & = 1 + \frac{p(k)}{100} \\
                r & = 1 + \frac{\frac{p(letna)}{k}}{100} \\
                r & = 1 + \frac{p(letna)}{k \cdot 100}
            \end{split}
        \end{equation}

        s tem pa je enačba za izračun glavnice po $k$ kapitalizacijskih obdobjih 
        ($k = k(letno) \cdot n$):
        \begin{equation}
            \begin{split}
                G_n & = G_0 \cdot r^k \\
                G_n & = G_0 \cdot (1 + \frac{p(na \: kapitalizacijsko \: obdobje)}{100 \cdot k})^k
            \end{split}
        \end{equation}

        \quad \\
        \quad \\

        \begin{longtable}{|c|c|c|S|S|}
            \hline
            \textbf{leto} & \textbf{četrtletje} & \textbf{$k$} & \textbf{letno} & \textbf{četrtletno} \\ \hline
            \endfirsthead
            \endhead
            0 & 0  & 0   & 10000.00 & 10000.00 \\ \hline \hline
            1 & 1  & 1   & 10000.00 & 10125.00 \\ \hline
              & 2  & 2   & 10000.00 & 10251.56 \\ \hline
              & 3  & 3   & 10000.00 & 10379.71 \\ \hline
              & 4  & 4   & 10500.00 & 10509.45 \\ \hline \hline
            2 & 1  & 5   & 10500.00 & 10640.82 \\ \hline 
              & 2  & 6   & 10500.00 & 10773.83 \\ \hline
              & 3  & 7   & 10500.00 & 10908.50 \\ \hline
              & 4  & 8   & 11025.00 & 11044.86 \\ \hline \hline
            3 & 1  & 9   & 11025.00 & 11182.92 \\ \hline 
              & 2  & 10  & 11025.00 & 11322.71 \\ \hline
              & 3  & 11  & 11025.00 & 11464.24 \\ \hline
              & 4  & 12  & 11576.25 & 11607.55 \\ \hline \hline
            4 & 1  & 13  & 11576.25 & 11752.64 \\ \hline 
              & 2  & 14  & 11576.25 & 11899.55 \\ \hline
              & 3  & 15  & 11576.25 & 12048.29 \\ \hline
              & 4  & 16  & 12155.06 & 12198.90 \\ \hline \hline
            5 & 1  & 17  & 12155.06 & 12351.38 \\ \hline 
              & 2  & 18  & 12155.06 & 12505.77 \\ \hline
              & 3  & 19  & 12155.06 & 12662.10 \\ \hline
              & 4  & 20  & 12762.82 & 12820.37 \\ \hline
        \end{longtable}

        Opazimo lahko, da smo z obrestovanjem na koncu vsakega četrtletja pridobili več obresti
        kot pri obrestovanju letno. Od tod izvira tudi Eulerjevo število $e$, ki predstavlja
        100 \% obrestno mero. Če imamo v enem letu neskončno kapitalizacijskih obdobji,
        predstavlja število $e$ obrestovalni faktor, če bi obresti računali letno.
        Izračunamo ga lahko po formuli: \hfill \cite{wiki:euler}
        \begin{equation}
            e = \lim_{k \to \infty} (1 + \frac{1}{k})^k
        \end{equation}

        \subsubsection{Konformna obrestna mera}
        Uporaba 
\section{Avtentični primer}






\newpage
\begingroup
\makeatletter
    \section{Viri in literatura}
    \nocite{*}
    \printbibliography[heading=none]
\makeatother
\endgroup
\newpage

\begin{samepage}
    \thispagestyle{empty}
    \section*{Izjava o avtorstvu}
    Izjavljam, da je seminarska naloga v celoti moje avtorsko delo, ki sem ga 
    izdelal samostojno s pomočjo navedene literature in pod vodstvom mentorja.

    \vfill
    
    \today \hfill Jaka Kovač, G 4. b
    
    \vspace{3 cm}
\end{samepage}

\end{document}
